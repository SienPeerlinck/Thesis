% !TeX root = thesis.tex
% Hierbij wordt de oplossing grondig geëvalueerd. Besteed hier voldoende aandacht aan! De volgende elementen kunnen deel uitmaken van de evaluatie:

% Aftoetsen van vereisten. Toon aan in welke mate voldaan is aan de vereisten die zijn opgesteld in hoofdstuk 2. Wijs op mogelijks conflicterende vereisten.

% Evalueer de sterktes en beperkingen van je oplossing. Reflecteer over de technologieën en methodologie die je hebt ingezet, en mogelijke aandachtspunten bij het inzetten van de technologieën in andere toepassingen.

% Uitbreidingen. Geef een overzicht van mogelijke uitbreidingen, en de complexiteit om deze uitbreidingen te realiseren.

% Vergelijk met bestaand werk. Toon aan op welke parameters je oplossing beter of minder goed scoort dan bestaand werk.
\chapter{Evaluation}\label{ch:evaluation}